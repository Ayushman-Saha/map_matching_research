%! TEX root = ./main.tex
\usepackage[linesnumbered,algoruled,boxed,lined]{algorithm2e}
\usepackage{complexity} % For symbols in complexity class.
%\usepackage{todonotes} % For todo notes
%\presetkeys{todonotes}{inline}{}
%\usepackage{geometry}
 %\geometry{
 %a4paper,
 %total={170mm,257mm},
 %left=20mm,
 %top=20mm,
 %}
 
\usepackage{tikz}
\usetikzlibrary{positioning,backgrounds,patterns,calc}

\usepackage{enumitem}

\usepackage[noadjust]{cite} % to sort references inline

\usepackage{thmtools}
\usepackage{thm-restate}
\declaretheorem[name=Theorem,numberwithin=section]{thm}


\usetikzlibrary{patterns}
\usepackage{hyperref}
 \hypersetup{%
  colorlinks=true,% hyperlinks will be black
  linkbordercolor=red,% hyperlink borders will be red% border style will be underline of width 1pt
  citecolor=green!70!black,
}
%\usepackage{cleveref}
%\usepackage{subcaption}
%\usepackage[a4paper, total={6.0in, 8.5in}]{geometry}
\usepackage{subcaption}
\captionsetup{compatibility=false}

\usepackage{enumitem}
 
 \usepackage{graphicx}
 \usepackage{amsmath}
 \usepackage{amsthm}

%\usepackage[dvipsnames]{xcolor}
\newcommand{\prelim}[1]{\textcolor{cyan}{#1}}
\newcommand{\tocheck}[1]{\textcolor{cyan}{#1}}
\newcommand{\todo}[1]{\textcolor{red}{\textbf{#1}}}
\newcommand{\blue}[1]{{\color{blue}{#1}}}
\newcommand{\red}[1]{{\color{red}[#1]}}
\newcommand{\dc}[1]{\textcolor{cyan}{#1 - DC}}
\newcommand{\ff}[1]{\textcolor{violet}{#1 - FF}}
\newcommand{\mm}[1]{\textcolor{green}{#1 - MM}}
\newcommand{\od}[1]{\textcolor{teal}{#1}}
\newcommand{\pt}[1]{\textcolor{purple}{#1 - PT}}
\newcommand{\SoPP}{\textsc{Shortest Path Partition}\xspace}
\newcommand{\dist}[2]{d\left(#1,#2\right)}

\newcommand{\partx}[1]{\mathcal{P}_X(#1)}
\newcommand{\partu}[1]{\mathcal{P}_U(#1)}
\newcommand{\rpartx}[1]{\mathcal{R}_X(#1)}
\newcommand{\VCLD}{$k$-VC LD-Set}
\newcommand{\LD}{\textsc{Locating-Dominating Set}\xspace}
\newcommand{\TCPB}{\textsc{Test Cover}\xspace}

\newcommand{\Rangle}{\rangle \hspace{-0.8mm}\rangle}
\newcommand{\Langle}{\langle \hspace{-0.8mm}\langle}
\newcommand{\test}{\mathcal{T}}
\newcommand{\rb}{\textsc{R-B Location}~}
\newcommand{\bigo}{\mathcal{O}}


\newcommand{\setrep}{\texttt{set-rep}}
\newcommand{\bitrep}{\texttt{bit-rep}}
\newcommand{\bit}{\texttt{bit}}

\newcommand{\dptw}{\texttt{opt}}

\newcommand{\ETH}{{\textsf{ETH}}}

\newcommand{\bbN}{{\mathbb N}}
\newcommand{\bbR}{{\mathbb R}}

\newcommand{\diam}{\textsf{diam}}
\newcommand{\tw}{\textsf{tw}}
\newcommand{\tl}{\textsf{tl}}
\newcommand{\ctw}{\textsf{ctw}}
\newcommand{\td}{\textsf{td}}
\newcommand{\fvs}{\textsf{fvs}}
\newcommand{\fes}{\textsf{fes}}
\newcommand{\vc}{\textsf{vc}}
\newcommand{\nd}{\textsf{nd}}
\newcommand{\cvd}{\textsf{cvd}}
\newcommand{\tc}{\textsf{tc}}
\newcommand{\dclique}{\textsf{dc}}
\newcommand{\bd}{\textsf{bd}}

\newcommand{\ld}{\gamma^{LD}}
\newcommand{\calA}{\mathcal{A}}
\newcommand{\calB}{\mathcal{B}}
\newcommand{\calC}{\mathcal{C}}
\newcommand{\dom}{\mathcal{D}}
\newcommand{\calE}{{\mathcal{E}}}
\newcommand{\calF}{\mathcal{F}}
\newcommand{\calG}{\mathcal{G}}
\newcommand{\calH}{{\mathcal H}}
\newcommand{\calI}{{\mathcal I}}
\newcommand{\calO}{\mathcal{O}}
\newcommand{\calP}{\mathcal{P}}
\newcommand{\calS}{\mathcal{S}}
\newcommand{\calT}{\mathcal{T}}
\newcommand{\calV}{\mathcal{V}}
\newcommand{\calW}{\mathcal{W}}
\newcommand{\calX}{\mathcal{X}}
\newcommand{\calY}{\mathcal{Y}}
\newcommand{\calZ}{\mathcal{Z}}

\newcommand{\true}{\texttt{True}}
\newcommand{\false}{\texttt{False}}

\newcommand{\POLY}{\textsf{poly}}
\newcommand{\Complete}{\textsf{Complete}}

\newcommand{\yes}{\textsc{Yes}}
\newcommand{\no}{\textsc{No}}
\newcommand{\YES}{\textsc{Yes}}
\newcommand{\NO}{\textsc{No}}

%
\newcommand{\id}{\mathcal{I}}
\newcommand{\un}{\mathcal{U}}
\newcommand{\triv}{{\rm Tr}}
\newcommand{\m}{\mathcal{M}}
\newcommand{\module}[2]{\mathcal{M}_{#1}(#2)}
%\newcommand{\p}{\mathcal{P}}
\newcommand{\x}{\mathcal{X}}
\newcommand{\scrP}{\mathscr{P}}
\newcommand{\s}{\mathcal{S}}
\newcommand{\q}{\mathcal{Q}}
%\newcommand{\bigo}{\mathcal{O}}
\newcommand{\ostar}{\mathcal{O}^\star}
\newcommand{\cell}[2]{{\rm C}_{#1}(#2)}
\newcommand{\nbd}[3]{N_{#1|#2}(#3)}
\newcommand{\cnbd}[3]{N_{#1|#2}[#3]}
\newcommand{\lset}{{\large \sc \texttt{L-Set}}}
\newcommand{\ldset}{{\large \sc \texttt{LD-Set}}}
\newcommand{\qlset}[1]{{\large \sc \texttt{L($#1$)-Set}}}
\newcommand{\qldset}[1]{{\large \sc \texttt{LD($#1$)-Set}}}
\newcommand{\rbqlset}[1]{{\large \sc \texttt{RB-L($#1$)-Set}}}
\newcommand{\rbqldset}[1]{{\large \sc \texttt{RB-LD($#1$)-Set}}}
\newcommand{\lc}{\texttt{opt}}
%\newcommand{\LD}{\textsc{Location-Domination}}
%\newcommand{\PP}{\textsc{Pre-Partitioned Red-Blue Test Cover}}

%\newtheorem{theorem}{Theorem}
%\newtheorem{lemma}{Lemma}
%\newtheorem{corollary}{Corollary}[section]
%\newtheorem{observation}{Observation}[section]
%\newtheorem{proposition}{Proposition}[section]
\newtheorem{reduction rule}[theorem]{Reduction Rule}
%\newtheorem{marking-scheme}{Marking Scheme}[section]
%\newtheorem{remark}{Remark}[section]

% custom labels for heavy definition environments

\makeatletter
\newcommand{\customlabel}[2]{%
\protected@write \@auxout {}{\string \newlabel {#1}{{#2}{}}}}
\makeatother


% ################ NEW ENVIROMENT STARTS ##########################
\newcommand{\defparproblem}[4]{
  \vspace{1mm}
\noindent\fbox{
  \begin{minipage}{0.96\textwidth}
  \begin{tabular*}{\textwidth}{@{\extracolsep{\fill}}lr} #1  & {\bf{Parameter:}} #3
\\ \end{tabular*}
  {\bf{Input:}} #2  \\
  {\bf{Question:}} #4
  \end{minipage}
  }
  \vspace{1mm}
}

\newcommand{\defparproblemoutput}[4]{
  \vspace{1mm}
\noindent\fbox{
  \begin{minipage}{0.96\textwidth}
  \begin{tabular*}{\textwidth}{@{\extracolsep{\fill}}lr} #1  & {\bf{Parameter:}} #3
\\ \end{tabular*}
  {\bf{Input:}} #2  \\
  {\bf{Output:}} #4
  \end{minipage}
  }
  \vspace{1mm}
}


\newcommand{\defproblem}[3]{
  \vspace{1mm}
\noindent\fbox{
  \begin{minipage}{0.96\textwidth}
  \begin{tabular*}{\textwidth}{@{\extracolsep{\fill}}lr} #1 \\ \end{tabular*}
  {\bf{Input:}} #2  \\
  {\bf{Question:}} #3
  \end{minipage}
  }
  \vspace{1mm}
}

\newcommand{\defproblemaux}[4]{
  \vspace{1mm}
\noindent\fbox{
  \begin{minipage}{0.96\textwidth}
  \begin{tabular*}{\textwidth}{@{\extracolsep{\fill}}lr} #1 \\ \end{tabular*}
  {\bf{Input:}} #2  \\
  {\bf{Question:}} #3\\
  {\bf{Auxiliary Graph:}} #4
  \end{minipage}
  }
  \vspace{1mm}
}

\newcommand{\defproblemout}[3]{
  \vspace{1mm}
\noindent\fbox{
  \begin{minipage}{0.96\textwidth}
  \begin{tabular*}{\textwidth}{@{\extracolsep{\fill}}lr} #1 \\ \end{tabular*}
  {\bf{Input:}} #2  \\
  {\bf{Output:}} #3
  \end{minipage}
  }
  \vspace{1mm}
}

\newcommand{\parnamedefn}[4]{
\begin{tabbing}
\name{#1} \\
\emph{Input:} \hspace{1cm} \= \parbox[t]{10cm}{#2} \\
\emph{Parameter:}            \> \parbox[t]{10cm}{#3} \\
\emph{Question:}             \> \parbox[t]{10cm}{#4} \\
% \emph{Input:} \hspace{1.2cm} \= \parbox[t]{9.5cm}{#2} \\
% \emph{Parameter:}            \> \parbox[t]{9.5cm}{#3} \\
% \emph{Question:}             \> \parbox[t]{9.5cm}{#4} \\
\end{tabbing}
\vspace{-0.5cm}
}

\newcommand{\decnamedefn}[3]{
  \begin{tabbing} #1\\
    \emph{Input:} \hspace{1.2cm} \= \parbox[t]{12cm}{#2} \\
    \emph{Question:}             \> \parbox[t]{12cm}{#3} \\
  \end{tabbing}
}
% ################ NEW ENVIROMENT ENDS ##########################
